
\label{sec:inference}

In this section we provide the conditional distributions required to Gibbs sample the HSLDA posterior distribution.  Note that, like in collapsed Gibbs samplers for LDA \cite{Griffiths04}, we have analytically marginalized out the parameters $\boldsymbol{\phi}_{1:K}$
and $\boldsymbol{\theta}_{1:D}.$   In the following $\mathbf{a}$ is the set of all auxiliary variables, $\mathbf{w}$ is the set of all words, $\boldsymbol\eta$ is the set of all regression coefficients, and  $\mathbf{z}_d\backslash z_{n,d}$ is the set $\mathbf{z}_d$ with element $z_{n,d}$ removed.
%
% In the Bayesian approach to statistical modeling,
%the primary task of inference is to find the posterior distribution
%over the unobserved parameters of the model. However, it is often
%possible and desirable to integrate over certain variables in the
%model, also known as collapsing. In our model, it will often be the
%case that the set of labels $\mathcal{L}$ is not fully observed for
%every document. We will define $\mathcal{L}_{d}$ to be the subset
%of labels which have been observed for document $d$. It is straightforward
%to integrate out the variables $a_{l^{\prime},d}$ and $y_{l^{\prime},d}$
%for $l^{\prime}\in\mathcal{L}\backslash\mathcal{L}_{d}$ from the
%full generative model. We can also integrate out the parameters $\boldsymbol{\phi}_{1:K}$
%and $\boldsymbol{\theta}_{1:D}$ as in \citet{Griffiths04}. Therefore,
%in our model the latent variables are $\mathbf{z}=\{z_{1:N_{d},d}\}_{d=1,\ldots,D},\boldsymbol\eta=\{\boldsymbol\eta_{l}\}_{l\in\mathcal{L}},\mathbf{a}=\{a_{l^{\prime},d}\}_{l^{\prime}\in\mathcal{L}_{d},d=1,\ldots,D},\boldsymbol\beta,\alpha,\alpha^{\prime}$
%and $\gamma$.
%
%The posterior distribution we seek cannot be solved in closed form.
%This is often the case in evaluating posterior distributions of non-trivial
%probabilistic models. We will appeal to one of the common methods
%for approximating posterior distributions in the face of intractable
%normalization factors: Markov chain Monte Carlo (MCMC) sampling. Since
%in this model it is possible to sample from the conditional distributions
%for all variables we will use the Gibbs sampling algorithm to obtain
%an approximation to this posterior. %In particular, we will derive a collapsed Gibbs sampler.
%Given an observation of a set of observed labels and a document, the posterior distribution for the latent variables is
%\begin{equation}
%p\left(\theta,z_{1:N},\phi_{1:K},\boldsymbol\eta_{i_{l,c}\in\mathcal{I}},a_{i_{l,c}\in\mathcal{I}},\boldsymbol\beta,\alpha,\alpha',\gamma\mid w_{1:N},y_{i_{l,c}\in\mathcal{I}};\sigma,\lambda\right)\label{eq:Posterior}\end{equation}
%\[
%=\frac{p\left(\theta,z_{1:N},\phi_{1:K},\boldsymbol\eta_{i_{l,c}\in\mathcal{I}},a_{i_{l,c}\in\mathcal{I}},\boldsymbol\beta,\alpha,\alpha',\gamma,w_{1:N},y_{i_{l,c}\in\mathcal{I}};\sigma,\lambda\right)}{\int_{\theta,\phi,a,\boldsymbol\eta,\alpha,\alpha',\boldsymbol\beta,\gamma}\sum_{z}p\left(\theta,z_{1:N},\phi_{1:K},\boldsymbol\eta_{i_{l,c}\in\mathcal{I}},a_{i_{l,c}\in\mathcal{I}},\boldsymbol\beta,\alpha,\alpha',\gamma,w_{1:N},y_{i_{l,c}\in\mathcal{I}};\sigma,\lambda\right)}\]
%
%
%
%\subsection{Gibbs Sampler}
%
%We derive a collapsed Gibbs sampler for this model by considering
%the individual conditional probability distributions for each of the
%unobserved variables. We use the notation $\mathbf{z}_{-(n,d)}$ to
%denote $\mathbf{z}_d\backslash z_{n,d}$. %, marginalizing
%$\theta_{1:D}$ and $\phi_{1:K}$. For details regarding collapsing
%in LDA models see \citet{Griffiths04}. 
%
%
%
%\subsection{$p(z_{n,d}\mid\mathbf{z}_{-(n,d)},\mathbf{a},\mathbf{w},\mathbf{\boldsymbol\eta},\alpha,\boldsymbol\beta,\gamma)$}
%
%First we consider the conditional distribution of the assignment variable
%for each word $n=1,\ldots,N_{d}$ in documents $d=1,\ldots,D$. The
%conditional distribution does not include $\theta_{1:D}$ and $\phi_{1:K}$
%because they have been integrated out as in the collapsed Gibbs sampler
%\citep{Griffiths04}. The conditional distribution of $z_{n,d}$ is
%proportional to the joint distribution of its markov blanket. %\begin{equation}
%p\left(z_{d,n}\mid\mathbf{z_{-\left(d,n\right)}},\mathbf{a},\mathbf{w},\mathbf{\boldsymbol\eta},\alpha,\boldsymbol\beta,\gamma\right)\propto p\left(z_{d,n},\mathbf{z_{-\left(d,n\right)}},\mathbf{a},\mathbf{w},\mathbf{\boldsymbol\eta},\alpha,\boldsymbol\beta,\gamma\right)\end{equation}
%Due to the factorization of this model, we can rewrite the joint distribution
%as the following:
%\begin{equation}
%p\left(z_{n,d}\mid\mathbf{z}_d\backslash z_{n,d},\mathbf{a},\mathbf{w},\mathbf{\boldsymbol\eta},\alpha,\boldsymbol\beta,\gamma\right)\propto\prod_{l\in\mathcal{L}_{d}}p\left(a_{l,d}\mid\mathbf{z},\boldsymbol\eta_{l}\right)p\left(z_{n,d}\ |\ \mathbf{z}_d\backslash z_{n,d},\mathbf{a},\mathbf{w},\alpha,\boldsymbol\beta,\gamma\right).\end{equation}
% The product is only over the subset of labels $\mathcal{L}_{d}$
%which have been observed for document $d$. By isolating terms that
%depend on $z_{n,d}$ and absorbing all other terms into a normalizing
%constant as in \citep{Griffiths04} we find 
\begin{eqnarray}
\lefteqn{p\left(z_{n,d}=k\mid\mathbf{z}_d\backslash z_{n,d},\mathbf{a},\mathbf{w},\mathbf{\boldsymbol\eta},\alpha,\boldsymbol\beta,\gamma\right)\propto}\nonumber \\
 & \hspace{2cm}\left(c_{\left(\cdot\right),d}^{k,-\left(n,d\right)}+\alpha\boldsymbol\beta_{k}\right)\frac{c_{w_{n,d},\left(\cdot\right)}^{k,-\left(n,d\right)}+\gamma}{\left(c_{\left(\cdot\right),\left(\cdot\right)}^{k,-\left(n,d\right)}+V\gamma\right)}\prod_{l\in\mathcal{L}_{d}}\exp\left\{ -\frac{\left(\bar{\mathbf{z}}_{d}^{T}\boldsymbol\eta_{l}-a_{l,d}\right)^{2}}{2}\right\} \label{eq:z-likelihood}\end{eqnarray}
 Here, $c_{v,d}^{k,-\left(n,d^{\prime}\right)}$ is the number
of words of type $v$ in document $d$ assigned to topic $k$ omitting
the $n$th word of document $d^{\prime}$. The $(\cdot)$
in the subscript means the count resulting from summing over the omitted
subscript variable.  Also $\mathcal{L}_{d}$ is the set of labels which are observed for document $d$.

%Given Equation \ref{eq:z-likelihood}, $p\left(z_{d,n}\mid\mathbf{z}_{-\left(d,n\right)},\mathbf{a},\mathbf{w},\mathbf{\boldsymbol\eta},\alpha,\boldsymbol\beta,\gamma\right)$
%can be sampled through enumeration. 

The conditional posterior distribution of the regression coefficients is given by 
\begin{equation}
p(\boldsymbol\eta_{l}\mid\mathbf{z},\mathbf{a},\sigma) = \mathcal{N}(\hat{\boldsymbol\mu}_{l},\hat{\mathbf{\Sigma}})\label{eqn:regression_param_conditional}
\end{equation}
%$\boldsymbol\eta_{l}$ for $l\in\mathcal{L}$. Given that $\boldsymbol\eta_{l}$
%and $a_{l,d}$ are distributed normally, the posterior distribution
%of $\boldsymbol\eta_{l}$ is normally distributed with mean $\hat{\boldsymbol\mu}_{l}$
%and covariance $\hat{\mathbf{\Sigma}}$ such that % (probably not the right place for this) We evaluated the model over various values of $\sigma$ where $\sigma=\left\{ 0.01,0.1,0.25,1,2\right\} $.
where
\begin{equation*}
\hat{\boldsymbol\mu}_{l}  =  \hat{\mathbf{\Sigma}}\left(\mathbf{1}\frac{\mu}{\sigma}+\bar{\mathbf{Z}}^{T}\mathbf{a}_{l}\right) \qquad \hat{\mathbf{\Sigma}}^{-1}  =  \mathbf{I}\sigma^{-1}+\bar{\mathbf{Z}}^{T}\bar{\mathbf{Z}}
.\end{equation*}
Here $\bar{\mathbf{Z}}$ is a $D\times K$ matrix
such that row $d$ of $\mathbf{\bar{Z}}$ is $\bar{\mathbf{z}}_{d}$, and $\mathbf{a}_{l}=[a_{l,1},a_{l,2},\ldots,a_{l,D}]^{T}$.  The simplicity of this conditional distribution follows from the choice of probit regression  \citep{Gelman}; the specific form of the update is a standard result from Bayesian normal  linear regression \citep{Gelman}. 
%p\left(\boldsymbol\eta_{i_{l,c}}\mid\mathbf{z}_{1:D},\mathbf{a};\sigma\right)=\mathcal{N}\left(\boldsymbol\eta_{i_{l,c}}\mid\hat{\mu}_{i},\hat{\Sigma}_{i}\right)\end{equation}
%\[
%\[
%\subsection{$p\left(a_{l,d}\mid\mathbf{z},\mathbf{Y},\mathbf{\boldsymbol\eta}\right)$}
%and \textmd{$p\left(y_{m,i}\mid\mathbf{a}\right)$}}
%The auxiliary variables $a_{l,d}$ must be sampled for documents $d=1,\ldots,D$
%and $l\in\mathcal{L}_{d}$. The conditional posterior distribution
%of 
That the conditional posterior distribution of $a_{l,d}$ is a truncated normal distribution \begin{equation}
p\left(a_{l,d,}\mid\mathbf{z},\mathbf{Y},\mathbf{\boldsymbol\eta}\right)\propto\frac{1}{\sqrt{2\pi}}\exp\left\{ -\frac{1}{2}\left(a_{l,d}-\boldsymbol\eta_{l}^{T}\mathbf{\bar{z}}_{d}\right)\right\} \mathbb{I}\left(a_{l,d}y_{l,d}>0\right).\label{eqn:a_l_d}\end{equation}
 is also a standard probit regression result \cite{gelman}.
% This conditional distribution can be sampled using an inverse CDF
%method. %However, if $y_{d,i_{l,c}}$ is unobserved then $a_{d,i_{l,c}}$must
%be sampled jointly with $y_{d,i_{l,c}}$ to ensure that the Markov
%chain is ergodic. Suppose that $a_{d,i_{l,c}}$ is sampled to have
%a negative value and $y_{d,i_{l,c}}$ is apporopriately sampled at
%-1. Although there exist valid states where $a_{d,i_{l,c}}>0$ and
%$y_{d,i_{l,c}}=1$, they will never be reached by such a Markov chain
%since $p\left(a_{d,i_{l,c}}<0\mid y_{d,i_{l,c}}=-1\right)=1$ and
%$p\left(y_{d,i_{l,c}}=-1\mid a_{d,i_{l,c}}<0\right)=1$. Therefore,
%to ensure ergodicity, $a_{d,i_{l,c}}$and $y_{d,i_{l,c}}$ must be
%sampled from the joint distribution as shown in Equation \ref{eq:Probit-Joint}.
%\begin{equation}
%p\left(a_{d,i_{l,c}},y_{d,i_{l,c}}\mid\mathbf{z},\mathbf{Y}_{-\left(d,i_{l,c}\right)},\mathbf{\boldsymbol\eta}\right)\propto p\left(y_{i_{l,c}}\mid\mathbf{a},\mathbf{y}_{-\left(l,c\right)}\right)p\left(a_{d,i_{l,c}}\mid\mathbf{z},\mathbf{Y},\mathbf{\boldsymbol\eta}\right)\end{equation}
%\begin{equation}
%p\left(y_{i_{l,c}}\mid\mathbf{a},\mathbf{y}_{-\left(l,c\right)}\right)=\delta\left(sign\left(a_{d,i_{l,c}}\right)=y_{i_{l,c}}\right)p\left(y_{i_{l,c}}\mid y_{parents_{l,c}}\right)\prod_{i_{\hat{l},\hat{c}}\in children_{l,c}}p\left(y_{i_{\hat{l},\hat{c}}}\mid y_{i_{l,c}}\right)\end{equation}
%\begin{equation}
%p\left(y_{i_{l,c}}=-1\mid y_{parent{}_{l,c}}\right)=\begin{cases}
%1, & y_{parent_{l,c}}=-1\\
%0.5, & y_{parent_{l,c}}=1\end{cases}\end{equation}
%\[
%p\left(a_{d,i_{l,c}},y_{d,i_{l,c}}\mid\mathbf{z},\mathbf{Y}_{-\left(d,i_{l,c}\right)},\mathbf{\boldsymbol\eta}\right)\]
%\begin{equation}
%=\begin{cases}
%\mathcal{N}\left(a_{d,i_{l,c}}\mid\bar{z}^{T}\boldsymbol\eta_{i_{l,c}},1\right)p\left(y_{d,i_{l,c}}\mid a_{d,i_{l,c}}\right), & y_{parent_{l,c}}=1,\forall y_{i_{\hat{l},\hat{c}}}\in y_{children_{l,c}},y_{i_{\hat{l},\hat{c}}}=-1\\
%trunc\mathcal{N}^{-}\left(a_{d,i_{l,c}}\mid\bar{z}^{T}\boldsymbol\eta_{i_{l,c}},1\right)\delta\left(y_{d,i_{l,c}}=-1\right), & y_{parent_{l,c}}=-1\\
%trunc\mathcal{N}^{+}\left(a_{d,i_{l,c}}\mid\bar{z}^{T}\boldsymbol\eta_{i_{l,c}},1\right)\delta\left(y_{d,i_{l,c}}=1\right), & \exists y_{i_{\hat{l},\hat{c}}}\in y_{children_{l,c}}\setminus y_{i_{\hat{l},\hat{c}}}=1\\
%0 & otherwise\end{cases}\end{equation}
%where $\mathbf{Y}_{-\left(d,i_{l,c}\right)}$ denotes all of the response
%variables excluding the response variable being sampled.
%
%
%
%\subsection{$p\left(\boldsymbol\beta\mid\mathbf{z},\alpha^{\prime},\alpha\right)$}

HSLDA departs from stock LDA in that we estimate a hierarchical Dirichlet prior over topic assignments (i.e.~$\boldsymbol\beta$ is a parameter in our model).  This has been shown to ADLER \citep{WallachMiMc2009}. 
%This prior
%shares many features with the hierarchical Dirichlet process and inference
%over this distribution proceeds in a very similar fashion.
%
Sampling $\boldsymbol\beta$ is done using the ``direct assignment''
method of \citet{TehJorBea2006} \begin{equation}
\boldsymbol\beta\mid\mathbf{z},\alpha^{\prime},\alpha\sim{\rm Dir}\left(m_{\left(\cdot\right),1}+\alpha^{\prime},m_{\left(\cdot\right),2}+\alpha^{\prime},\ldots,m_{\left(\cdot\right),K}+\alpha^{\prime}\right)\end{equation}
where $m_{d,k}$ are ADLER
 \begin{equation}
p\left(m_{d,k}=m\mid\mathbf{z},\mathbf{m}_{-\left(d,k\right)},\boldsymbol\beta\right)=\frac{\Gamma\left(\alpha\boldsymbol\beta_{k}\right)}{\Gamma\left(\alpha\boldsymbol\beta_{k}+n_{d,k}\right)}s\left(n_{d,k},m\right)\left(\alpha\boldsymbol\beta_{k}\right)^{m}\end{equation}
 where $s\left(n,m\right)$ represents stirling numbers of the first
kind.  ADLER check this.  I suspect that some of the n's need some summing.


%\subsection{$p\left(\alpha\right)$, $p\left(\alpha'\right)$, $p\left(\gamma\right)$}

The hyperparameters $\alpha$, $\alpha^{\prime}$, and $\gamma$ are
%given broad ${\rm Gamma}(1,1000)$ prior distributions and 
sampled
using Metropolis-Hastings. 

