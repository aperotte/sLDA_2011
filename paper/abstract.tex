We introduce hierarchically supervised latent Dirchlet allocation (HSLDA), a model for hierarchically and multiply labeled bag-of-words data.  Out-of-sample label prediction is the primary goal of this work; however, improved dimensionality reduction is also of interest.  We define a model that uses probit regressors on a conditionally dependent label hierarchy tied to latent Dirichlet allocation (LDA) in the same manner as was done in supervised LDA (SLDA) prior art.  We find that the additional supervision signal that comes from multiple, hierarchically constrained labels substantially improves out-of-sample label prediction in medical document labeling and product categorization tasks.  Additionally, held-out likelihood 

%Topic models are unsupervised models well suited to data exploration and information retrieval.  Supervised topic models use side-information, labels and the like, to improve the learned representations.

%The benefits of supervision in topic modeling 
%Current medical record keeping technology relies heavily upon human
%capacity to effectively summarize and infer information from free-text
%physician notes. We propose a novel method to suggest diagnostic code
%assignment for patient visits, based upon narrative medical notes.
%We applied a supervised latent Dirichlet allocation model to a corpora
%of free-text medical notes from NewYork - Presbyterian Hospital to
%infer a set of specific ICD-9 codes for each patient note. Evaluation
%of the predictions were conducted by comparison to a gold-standard
%set of ICD-9s assigned to a set of patient notes. 