We introduce hierarchically supervised latent Dirchlet allocation (HSLDA), a model for hierarchically and multiply-labeled bag-of-words data.  Out-of-sample label prediction is the primary goal of this work; however, improved dimensionality reduction is also of interest.  We define a model that uses probit regressors on a conditionally dependent label hierarchy tied to latent Dirichlet allocation (LDA).  We find that the additional signal that comes from multiple, hierarchically constrained labels substantially improves out-of-sample label prediction in comparison to supervised LDA approaches that don't utilize information derived from the structure of the label space.   We demonstrate HSLDA on large-scale data from medical document labeling and retail product categorization tasks and show both improved label prediction performance and show evidence that the learned topic model improves as a result of using this signal too.  

%Topic models are unsupervised models well suited to data exploration and information retrieval.  Supervised topic models use side-information, labels and the like, to improve the learned representations.

%The benefits of supervision in topic modeling 
%Current medical record keeping technology relies heavily upon human
%capacity to effectively summarize and infer information from free-text
%physician notes. We propose a novel method to suggest diagnostic code
%assignment for patient visits, based upon narrative medical notes.
%We applied a supervised latent Dirichlet allocation model to a corpora
%of free-text medical notes from NewYork - Presbyterian Hospital to
%infer a set of specific ICD-9 codes for each patient note. Evaluation
%of the predictions were conducted by comparison to a gold-standard
%set of ICD-9s assigned to a set of patient notes. 