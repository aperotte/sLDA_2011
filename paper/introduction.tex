There exist surprisingly many sources of unstructured text data that have been partially or completely categorized by human editors.  Examples include hierarchical directories of webpages \citep{DMOZ}, large hierarchically annotated product catalogs (e.g.~\citep{AMAZON} as available from \citep{SNAP}), manually annotated patient medical records, and many more.  In this work we show how to combine these two sources of information in a single model that allows us to, amongst other things, automatically annotate and/or categorize new text documents, effectively inserting them into the category

\begin{itemize}
\item Benefits of combining human categorization information into ``topic models''
\item LDA solved free text
\item supervised LDA improves LDA (extra info) and allows new inference (predict links, etc.)
\item amazon, freshdirect, netflix, dmoz, pandora (music genome)
\end{itemize}


% Informatics journal paper
%
% Despite the growing emphasis on meaningful
%use of technology in medicine, many aspects of medical record-keeping
%remain a manual process. Diagnostic coding for billing and insurance
%purposes is often handled by professional medical coders who must
%explore a patient's extensive clinical record before assigning the
%proper codes. So while electronic health records (EHRs) should be
%adopted by most medical institutions within the next several years,
%largely due to the provisions of HITECH under the American Recovery
%and Reinvestment Act \citep{Blumenthal2009}, there has been little
%movement forward in automating medical coding.

In this paper we describe the use of a topic model based on supervised
latent Dirichlet allocation (sLDA) to identify topics within narrative
discharge summaries and to automate the assignment of diagnostic codes,
specifically International Classification of Disease 9th Revision,
Clinical Modification (ICD-9-CM) codes. There are a number of advantages
to this approach. First, manually coding diagnoses is a time-consuming
and notoriously unreliable process. Many diagnoses are omitted in
the final record, and a high error rate is found even in the principal
diagnoses \citep{Surjan1999}.



An automated process would ideally produce a more complete and accurate
diagnosis lists. Also, this model will reveal information about the
medical records themselves. For example, we may gain an understanding
of what a specific code actually means in terms of clinical narratives.
Similarly, viewing the distribution of topics over discharge summaries
may reveal information about the latent structure of clinician documentation.
Lastly, the sLDA model would provide a novel approach to dealing with
the problem of high dimensionality when representing narrative text
in a vector space specifically by reducing dimensions from an entire
vocabulary of potentially tens of thousands of words to a set of several
dozen topics.

