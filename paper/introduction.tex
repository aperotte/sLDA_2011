There exist many sources of unstructured data that have been partially or completely categorized by human editors.  In this paper we focus on unstructured textual data that has been, at least in part, manually categorized.  Examples include but are not limited to webpages and curated hierarchical directories of the same \citep{DMOZ}, product descriptions and catalogs (e.g.~\citep{AMAZON} as available from \citep{SNAP}), and patient hospital treatment transcripts and  codes applied to them for bookkeeping and insurance purposes (e.g.~hospital discharge records with International Classification of Disease 9th Revision,
Clinical Modification (ICD-9-CM) codes assigned \cite{ICD9}).  In this work we show how to combine these two sources of information using a single model that allows one to automatically categorize new text documents, suggest labels that might be inaccurate, compute improved similarities between documents for information retrieval purposes, and more.  The models and techniques that we develop in this paper are applicable in other domains as well, namely, any unstructured representations of data that have been hierarchically classified (e.g.~image catalogs with bag-of-feature image representations). 

%In this work we extend supervised latent Dirichlet allocation (sLDA) \cite{BleiMcAuliffe2008} to take advantage of hierarchical supervision.  sLDA is latent Dirichlet allocation (LDA) \cite{Blei2003} augmented with per document ``supervision'';  often taking the form of a single numerical or categorical ``label.''  More generally this supervision is just extra per document data;  for instance its quality or relevance (e.g.~online review scores), marks given to written work (e.g.~essay grades), or the number of times a web page is linked.  These labels are usually generatively modeled as having been conditionally drawn from some distribution that depends on the document-specific topic mixture.  It has been demonstrated that the signal provided by such supervision can result in better, task-specific document models and can also lead to good label prediction for out-of-sample data \cite{BleiMcAuliffe2008}.

Our main contribution is to show how to utilize supervision in the form of  hierarchical and (often) multiple labelings in a similar manner.   Consider web retail data.  Web retailers often have both a browse-able product hierarchy and free-text descriptions for all products they sell.   The situation of each product in a product hierarchy (often multiply situated) constitutes a multiple, hierarchical labeling of the free-text product descriptions.  We hypothesize that such hierarchical labels should, at least in theory, provide better supervision than the simpler unstructured labels previously considered.  Results from applying our model to both medical record and web retail data suggests that this is likely the case.  In particular, we observe gains in our primary goal of out-of-sample label prediction that result specifically from leveraging hierarchical supervision. 

The remainder of this paper is as follows. Section~\ref{sec:model}
introduces hierarchically supervised LDA (HSLDA), while
Section~\ref{sec:inference} details a sampling approach to inference in HSLDA. 
Section~\ref{sec:related_work} reviews related work, and
Section~\ref{sec:experiments} shows results from applying HSLDA to health care
and web retail data.  

%In this paper we describe the use of a topic model based on supervised
%latent Dirichlet allocation (sLDA) to identify topics within narrative
%discharge summaries and to automate the assignment of diagnostic codes,
%specifically International Classification of Disease 9th Revision,
%Clinical Modification (ICD-9-CM) codes. There are a number of advantages
%to this approach. First, manually coding diagnoses is a time-consuming
%and notoriously unreliable process. Many diagnoses are omitted in
%the final record, and a high error rate is found even in the principal
%diagnoses \citep{Surjan1999}.

%Our main contribution is to show how to utilize supervision in the form of
%hierarchical and (often) multiple labelings in a similar manner. Consider web
%retail data. Web retailers often have both a browse-able product hierarchy and
%free-text descriptions for all products they sell. The situation of each
%product in a product hierarchy (often multiply situated) constitutes a
%multiple, hierarchical labeling of the free-text product descriptions. We
%hypothesize that such hierarchical labels should, at least in theory, provide
%better supervision than the simpler unstructured labels previously considered.
%Results from applying our model to both medical record and web retail data
%suggests that this is likely the case. In particular, we observe gains in our
%primary goal of out-of-sample label prediction that result specifically from
%leveraging hierarchical supervision. 

%\begin{itemize}
%\item Benefits of combining human categorization information into ``topic models''
%\item LDA solved free text
%\item supervised LDA improves LDA (extra info) and allows new inference (predict links, etc.)
%\item amazon, freshdirect, netflix, dmoz, pandora (music genome)
%\end{itemize}


% Informatics journal paper
%
% Despite the growing emphasis on meaningful
%use of technology in medicine, many aspects of medical record-keeping
%remain a manual process. Diagnostic coding for billing and insurance
%purposes is often handled by professional medical coders who must
%explore a patient's extensive clinical record before assigning the
%proper codes. So while electronic health records (EHRs) should be
%adopted by most medical institutions within the next several years,
%largely due to the provisions of HITECH under the American Recovery
%and Reinvestment Act \citep{Blumenthal2009}, there has been little
%movement forward in automating medical coding.

%An automated process would ideally produce a more complete and accurate
%diagnosis lists. Also, this model will reveal information about the
%medical records themselves. For example, we may gain an understanding
%of what a specific code actually means in terms of clinical narratives.
%Similarly, viewing the distribution of topics over discharge summaries
%may reveal information about the latent structure of clinician documentation.
%Lastly, the sLDA model would provide a novel approach to dealing with
%the problem of high dimensionality when representing narrative text
%in a vector space specifically by reducing dimensions from an entire
%vocabulary of potentially tens of thousands of words to a set of several
%dozen topics.

