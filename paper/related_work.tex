While there has been much work in multi-label classification of text and
text modeling in general, we focus here on topic modeling approaches.
% Noemie: do we need that much text on LDA?
Latent Dirichlet allocation (LDA) is a generative probabilistic model which
represents documents as a mixed-membership bag of word. Each document is
represented as a collection of words, generated from a set of topic assignments
(one for each word), where each topic assignment is drawn from a distribution
over topics~\citep{Blei2003}. sLDA is latent Dirichlet allocation (LDA)
\cite{Blei2003} augmented with per-document labeling, often taking the form of
a single numerical or categorical label. Examples of labels include rating
associated with an online  eview, grades for an essay, and number of times a
webpage is linked. This approach has been shown to outperform both LASSO
(L1 regularized least squares regression) and LDA followed by least
squares regression~\cite{BleiMcAuliffe2008}.

% Noemie: check out what I wrote here, because I am not really sure this is
% correct. 
There have been several models that incorporate both latent models of text and
some form of
supervision~\citep{Ramage2009,DiscLDA,wangbleifeifei08,RelationalLDA}. One set 
of models that are particularly relevant to HSLDA are Chang and Blei's
hierarchical models for document networks (Relational Topic Models). In that
family of models, they encountered a similar scenario where an unselected label 
does not always indicate absence. In hierarchical labels, this phenomenon is
even more pervasive -- there are no explicit negative labels, but it is also
unclear how to treat the parents of selected labels. Like in the work of Chang
and Blei, we employ regularization to account for the lack of negative
labeling. In our experiments, we look at the impact of assigning positive and
negative instances to the ancestors of selected labels.
