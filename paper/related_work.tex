In this work we extend supervised latent Dirichlet allocation (sLDA)
\cite{BleiMcAuliffe2008} to take advantage of hierarchical supervision. sLDA
is latent Dirichlet allocation (LDA) \cite{Blei2003} augmented with per
document ``supervision,'' often taking the form of a single numerical or
categorical label. It has been demonstrated that the signal provided by such
supervision can result in better, task-specific document models and can also
lead to good label prediction for out-of-sample data \cite{BleiMcAuliffe2008}.
It also has been demonstrated that sLDA has been shown to outperform both LASSO
(L1 regularized least squares regression) and LDA followed by least squares
regression~\cite{BleiMcAuliffe2008}. sLDA can be applied to data of the type
we consider in this paper; however, doing so requires ignoring the hierarchical
dependencies amongst the labels. In Section~\ref{sec:experiments} we constrast
HSLDA with sLDA applied in this way. 

%While there has been much work in multi-label classification of text and
%text modeling in general, we focus here on topic modeling approaches.
%Latent Dirichlet allocation (LDA) is a generative probabilistic model which
%represents documents as a mixed-membership bag of word. Each document is
%represented as a collection of words, generated from a set of topic assignments
%(one for each word), where each topic assignment is drawn from a distribution
%over topics~\citep{Blei2003}. sLDA is latent Dirichlet allocation (LDA)
%\cite{Blei2003} augmented with per-document labeling, often taking the form of
%a single numerical or categorical label. Examples of labels include ratings
%associated with online reviews, grades for essays, and the number of times a
%webpage is linked. This approach has been shown to outperform both LASSO
%(L1 regularized least squares regression) and LDA followed by least
%squares regression~\cite{BleiMcAuliffe2008}.

%Latent dirichlet allocation (LDA) is a generative
%probabilistic model of corpora that represents documents as a mixed
%membership bag-of-words. Also known as topic models, these models
%infer the latent structure, or topics, of documents in a corpus. Each
%document is represented as a collection of words, generated from a
%set of topic assignments (one for each word), where each topic assignment
%is drawn from a distribution over topics \citep{Blei2003}.

%Supervised latent Dirichlet allocation (sLDA) builds on LDA by incorporating
%supervision in the form of an observed exponential family response 
%variable per document. As a result, sLDA
%infers topics such that the model predicts the response variable while
%improing word likelihood.  This approach has been shown to outperform both LASSO
%(L1 regularized least squares regression) and LDA followed by least
%squares regression \citep{BleiMcAuliffe2008}.

Other models that incorporate LDA and supervision include
LabeledLDA\citep{Ramage2009} and DiscLDA\citep{DiscLDA}.  Various applications of these models to 
computer vision and document networks have been explored
\citep{wangbleifeifei08,RelationalLDA} .
None of these models, however, leverage dependency structure in the label space.

In other work, researchers have classified documents into a hierarchy (a closely related task) with naive Bayes classifiers
and support vector machines. Most of this work has been demonstrated on relatively
small datasets, small label spaces, and has focused on single label classification without
a model of documents such as LDA~\citep{mccallum99building,Dumais2000HCW,Kollerilprints291,Chakrabarti1998SFS}.

%One set of models that are particularly relevant to HSLDA are Chang
%and Blei's hierarchical models for document networks (Relational Topic
%Models). In that family of models, they encountered a similar scenario
%where the lack of a link did not truly indicate absence. In hierarchically
%labeled data, negative labels are uncommon and the lack of a label
%in the hierarchy is not equivalent to a negative label. Therefore,
%as in the work of Chang and Blei, we employ regularization to account
%for the lack of negative labels. This will be discussed further in

% Noemie: check out what I wrote here, because I am not really sure this is
% correct. 
%There have been several models that incorporate both latent models of text and
%some form of
%supervision~\citep{Ramage2009,DiscLDA,wangbleifeifei08,RelationalLDA}. One set 
%of models that are particularly relevant to HSLDA are Chang and Blei's
%hierarchical models for document networks (Relational Topic Models). In that
%family of models, they encountered a similar scenario where an unselected label 
%does not always indicate absence. In hierarchical labels, this phenomenon is
%even more pervasive -- there are no explicit negative labels, but it is also
%unclear how to treat the parents of selected labels. Like in the work of Chang
%and Blei, we employ regularization to account for the lack of negative
%labeling. In our experiments, we look at the impact of assigning positive and
%negative instances to the ancestors of selected labels.
