Latent dirichlet allocation (LDA) is a generative
probabilistic model of corpora that represents documents as a mixed
membership bag-of-words. Also known as topic models, these models
infer the latent structure, or topics, of documents in a corpus. Each
document is represented as a collection of words, generated from a
set of topic assignments (one for each word), where each topic assignment
is drawn from a distribution over topics \citep{Blei2003}.

Supervised latent Dirichlet allocation (sLDA) builds on LDA by incorporating
supervision in the form of an observed exponential family response 
variable per document. As a result, sLDA
infers topics such that the model predicts the response variable while
improing word likelihood.  This approach has been shown to outperform both LASSO
(L1 regularized least squares regression) and LDA followed by least
squares regression \citep{BleiMcAuliffe2008}.

Other models that incorporate LDA and supervision include
LabeledLDA\citep{Ramage2009}, DiscLDA\citep{DiscLDA}, and other models applied to 
computer vision and document networks\citep{wangbleifeifei08,RelationalLDA}. 
These models, however, do not implement contraints on the label space.

In other work, researchers have classified documents in a hierarchy with naive Bayes classifiers
and support vector machines. Most of this work has been demonstrated on relatively
small datasets, small label spaces and has focused on single label classification without
a model of documents such as LDA\citep{mccallum99building,Dumais2000HCW,Kollerilprints291,Chakrabarti1998SFS}.

%One set of models that are particularly relevant to HSLDA are Chang
%and Blei's hierarchical models for document networks (Relational Topic
%Models). In that family of models, they encountered a similar scenario
%where the lack of a link did not truly indicate absence. In hierarchically
%labeled data, negative labels are uncommon and the lack of a label
%in the hierarchy is not equivalent to a negative label. Therefore,
%as in the work of Chang and Blei, we employ regularization to account
%for the lack of negative labels. This will be discussed further in

% move to related work?
%While there have been some attempts to automatically classify groups
%of patients as a potential preliminary step to ICD-9 code assignment
%\citep{Ruch2008,FreitasJunior2006,RibeiroNeto2001,Brown2006}, fully
%automatic assignment of ICD-9 codes to medical text became a more
%prevalent research topic only in the last few years. A subset of earlier
%work proposed various methods on small corpora, based on a few specific
%diseases \citep{Rao2003} but the most recent and promising work on
%the subject was inspired by the 2007 Medical NLP Challenge: {}``International
%Challenge: Classifying Clinical Free Text Using Natural Language Processing\textquotedblright{}
%(website). Most of the classification strategies included word matching
%and rule-based algorithms. \citep{Goldstein2007,Crammer2007,Farkas2008}.
%The data set given to the participants consisted only of documents
%that were 1-2 lines each and all of the documents were radiology reports
%- clearly limiting the scope of potential ICD-9 codes which could
%be assigned. The only paper which has attempted to work with a document
%scope as large as ours was the 2008 Lita et al publication \citep{Lita2008}.
%Lita proposed support vector machine and bayesian ridge regression
%methods to assign appropriate labels to the documents but did not
%utilize the ICD-9 hierarchy to leverage more comprehensive predictions.
