\documentclass{article}
\usepackage{nips10submit_e,times}
%\documentstyle[nips07submit_09,times]{article}
\usepackage[square,numbers]{natbib}
\usepackage{amsmath, epsfig}
\usepackage{amsfonts}
\usepackage{subfigure}
\usepackage{graphicx}
\usepackage{amsfonts}
\usepackage{algorithm}
\usepackage{algorithmic}
\usepackage{easybmat}
\usepackage{footmisc}
\usepackage{lscape}
\renewcommand\algorithmiccomment[1]{// \textit{#1}}
%
\newcommand{\ignore}[1]{}
\newcommand{\comment}[1]{}
\DeclareMathOperator*{\argmax}{arg\,max}

\title{Supervised Topic Modeling in Clinical Text}

\author{
Perotte A\hspace{1cm} Li Y \hspace{1cm} Pivovarov R \hspace{1cm} Weiskopf N\hspace{1cm} Wood F\\
Columbia University, New York, NY 10027, USA \\
\texttt{\{\{ajp9009, yil7003, rip7002, ngw7001\}@dbmi., fwood@stat.\}columbia.edu}
%\texttt{pfau@neurotheory.columbia.edu} 
%\texttt{\{bartlett,fwood\}@stat.columbia.edu} 
}



% The \author macro works with any number of authors. There are two commands
% used to separate the names and addresses of multiple authors: \And and \AND.
%
% Using \And between authors leaves it to \LaTeX{} to determine where to break
% the lines. Using \AND forces a linebreak at that point. So, if \LaTeX{}
% puts 3 of 4 authors names on the first line, and the last on the second
% line, try using \AND instead of \And before the third author name.

\newcommand{\fix}{\marginpar{FIX}}
\newcommand{\new}{\marginpar{NEW}}
\newcommand{\X}{\mathcal{X}}


\nipsfinalcopy

\begin{document}

\maketitle

\begin{abstract}
Current medical record keeping technology relies heavily upon human capacity to effectively summarize and infer information from free-text physician notes.  We propose a novel method to suggest diagnostic code assignment for patient visits, based upon narrative medical notes.  We applied a supervised latent Dirichlet allocation model to a corpora of free-text medical notes from NewYork - Presbyterian Hospital to infer a set of specific ICD-9 codes for each patient note.  Evaluation of the predictions were conducted by comparison to a gold-standard set of ICD-9s assigned to a set of patient notes.
\end{abstract}


\section{Introduction}
\label{sec:introduction}
Despite the growing emphasis on meaningful use of technology in medicine, many aspects of medical record-keeping remain a manual process.  Diagnostic coding for billing and insurance purposes is often handled by professional medical coders who must explore a patient's extensive clinical record before assigning the proper codes.  So while electronic health records (EHRs) should be adopted by most medical institutions within the next several years, largely due to the provisions of HITECH under the American Recovery and Reinvestment Act \citep{Blumenthal2009}, there has been little movement forward in automating medical coding.  

In this paper we describe the use of a topic model based on supervised latent Dirichlet allocation (sLDA) to identify topics within narrative discharge summaries and to automate the assignment of diagnostic codes, specifically International Classification of Disease 9th Revision, Clinical Modification (ICD-9-CM) codes.  There are a number of advantages to this approach.  First, manually coding diagnoses is a time-consuming and notoriously unreliable process.  Many diagnoses are omitted in the final record, and a high error rate is found even in the principal diagnoses \citep{Surjan1999}.

While there have been some attempts to automatically classify groups of patients as a potential preliminary step to ICD-9 code assignment \citep{Ruch2008, FreitasJunior2006, RibeiroNeto2001, Brown2006}, fully automatic assignment of ICD-9 codes to medical text became a more prevalent research topic only in the last few years. A subset of earlier work proposed various methods on small corpora, based on a few specific diseases \citep{Rao2003} but the most recent and promising work on the subject was inspired by the 2007 Medical NLP Challenge: “International Challenge: Classifying Clinical Free Text Using Natural Language Processing” (website). Most of the classification strategies included word matching and rule-based algorithms. \citep{Goldstein2007, Crammer2007, Farkas2008}.  The data set given to the participants consisted only of documents that were 1-2 lines each and all of the documents were radiology reports - clearly limiting the scope of potential ICD-9 codes which could be assigned. The only paper which has attempted to work with a document scope as large as ours was the 2008 Lita et al publication \citep{Lita2008}.  Lita proposed support vector machine and bayesian ridge regression methods to assign appropriate labels to the documents but did not utilize the ICD-9 hierarchy to leverage more comprehensive predictions.

An automated process would ideally produce a more complete and accurate diagnosis lists.  Also, this model will reveal information about the medical records themselves.  For example, we may gain an understanding of what a specific code actually means in terms of clinical narratives.  Similarly, viewing the distribution of topics over discharge summaries may reveal information about the latent structure of clinician documentation.  Lastly, the sLDA model would provide a novel approach to dealing with the problem of high dimensionality when representing narrative text in a vector space specifically by reducing dimensions from an entire vocabulary of potentially tens of thousands of words to a set of several dozen topics.  


\section{Methods}

\subsection{Data}
Our data set was gathered from the clinical data warehouse of NewYork - Presbyterian Hospital. The data consisted of free-text discharge summaries and their respective ICD-9 codes. A discharge summary is a clinical report prepared by a physician or other health professional at the conclusion of a hospital stay or series of treatments. The note outlines the patient’s chief complaint, diagnostic findings, therapy administered, patient’s response to the chosen therapy, the treatment plan and the recommendations upon discharge. The ICD-9 codes used to structure the discharge summary data are part of a controlled terminology which is the international standard diagnostic classification for epidemiological, health management, and clinical purposes (http://www.who.int/classifications/icd/en/). The codes are classified in a rooted-tree structure, with each edge representing an is-a relationship between parent and child, such that the parent diagnosis subsumes the child diagnosis.  In the hospital, ICD-9 codes are generated manually by trained medical coders, who review all the information in the discharge summary.  For the purposes of sLDA, ICD-9 codes will be used as labels for discharge summaries.

We worked within the guidelines of the Health Insurance Portability and Accountability Act (HIPAA), which protects patient privacy and the security of potentially identifying medical material, known as personal health information (PHI).  HIPAA covers any information within a medical record that was created, used, or disclosed during the course of providing a health care service and that can be used to identify an individual.  Before beginning data processing, we generated a PHI-free dataset (see Data Pre-processing below).

\subsection{Pre-Processing}
Patient discharge summaries and their associated ICD-9 diagnoses are stored in two different places in the NewYork - Presbyterian data warehouse, and so had to be linked together before being fed into the sLDA algorithm.  Each discharge note and set of diagnoses were assigned a patient unique identifier (PUID) and a visit unique identifier (VUID), allowing the two types of data to be linked.

Natural Language Processsing (NLP) techniques were used to process the free-text discharge summaries.  First, the Natural Language Toolkit (http://www.nltk.org/) was used to tokenize the text.  Next, feature selection was performed using a term frequency - inverse document frequency (TF-IDF) algorithm on the entire document set and sorting the words by their TF-IDF values.  The top 10,000 words were manually evaluated to eliminate all potentially identifying information.  Finally, each discharge summary was converted to a bag-of-words, listing the frequencies of the remaining, free of protected health information, top 10,000 words.

Preparation of the diagnostic codes involved inference over the ICD-9 hierarchy.  The is-a relationships of the hierarchy allowed us to make two important assumptions.  First, if a diagnosis was observed to be present, all of its ancestors could be assumed to be present as well (e.g., if a patient had malignant hypertension, it could be assumed that they also had essential hypertension.  Second, if a diagnosis was observed to be absent, it could be assumed that all of its descendants were also absent (e.g. if a patient did not have essential hypertension, it could be assumed that they did not have malignant hypertension).  Unfortunately, ICD-9 code observations never include observations of disease absence.  ICD-9 codes are only documented when the condition is observed to be present.  Additionally, ICD-9 codes are known to have relatively low sensitivity; conditions that are present are often not documented in a set of ICD-9 codes (Surjan, 1999). Given these facts, we made the following assumptions regarding each visit: recorded diagnoses and their ancestors were labeled as true; diagnoses that were observed at some time for a patient but not at the current visit were labeled as unobserved; and diagnoses that had never been listed for a patient were labeled as false for all of that patient’s visits. This last assumption captures the belief that parts of the ICD-9 hierarchy that are never observed for a particular patient are almost certain to be false.  Additionally, for computational purposes, we decided not to include an ICD-9 code at all if neither it nor one of its descendants had been assigned to a patient in any of the records in our dataset.

\subsection{Supervised Topic Models}
Latent dirichlet allocation (LDA) is a generative probabilistic model of corpora that represents documents as a mixed membership bag-of-words. Also known as topic models, these models infer the latent structure, or topics, of documents in a corpus. Each document is represented as a collection of words, generated from a set of topic assignments (one for each word), where each topic assignment is drawn from a distribution over topics \citep{Blei2003}.

This model, supervised latent dirichlet allocation (sLDA), builds on LDA by incorporating an exponential family response variable. Although there are many models for making predictions based on free text, sLDA is unique in that it is a generative model, it represents documents as a mixed-membership, and constrains the inference of the latent structure of the documents by its predictability of the response variable.  In other words, sLDA infers topics such that the model is capable of a high predictive likelihood for words in a document and the response variable associated with a document.  This approach has been shown to outperform both LASSO (L1 regularized least squares regression) and LDA followed by least squares regression \citep{BleiMcAuliffe2008}.

Here, we augment the sLDA model such that the supervised signal is distribution over the ICD-9 code tree, which is an is-a hierarchy \citep{ICD9NCBO}. An is-a hierarchy is represented by the tree data structure where each node has only a single parent and nodes cannot be parents of ancestors (ie. there are no loops). In this particular case, the ICD-9 code hierarchy is also partially a prefix trie where the labels for certain nodes are prefixes for child nodes. Given that this rule does not apply to all nodes in the hierarchy, we did not use this feature to determine the structure of the hierarchy.  Instead we acquired a dataset that explicitly defined the relationships between the nodes of the hierarchy \citep{ICD9NCBO}. In documentation of ICD-9 codes for billing purposes, only a subset of the nodes can be used, however the nodes higher in the hierarchy contain semantic information about the categories of codes that are their descendants. For this reason, we included these nodes in our model.

\subsection{Generative Model}

%
\begin{figure}[h]
%tbp] %  figure placement: here, top, bottom, or page
 \centering \includegraphics[width=4in]{model_1} \includegraphics[width=3in]{model_2}
\caption{adapted sLDA model}


\label{fig:example} 
\end{figure}


Given the number of topics, K, the global prior over topic proportions,
$\alpha'$, and the prior over topics, $\gamma$, the generative process
for documents and responses is as follows: 
\begin{enumerate}
\item For each topic:

\begin{enumerate}
\item Draw a distribution over words $\beta_{k}\sim Dir\left(u,\gamma\right)$
\end{enumerate}
\item For each ICD9 Code:

\begin{enumerate}
\item Draw regression coefficient $\eta_{i}\mid\mu,\sigma\sim\mathcal{N}\left(\mu,\sigma\right)$ 
\end{enumerate}
\item Draw a prior over topic proportions $m\mid\alpha'\sim Dir\left(u,\alpha'\right)$ 
\item For each document:

\begin{enumerate}
\item Draw topic proportions $\theta_{d}\mid\alpha\sim Dir\left(m,\alpha\right)$ 
\item For each word:

\begin{enumerate}
\item Draw topic assignment $z_{n,d}\mid\theta_{d}\sim Mult\left(\theta_{d}\right)$ 
\item Draw word $w_{n,d}\mid z_{n,d},\beta_{1:K}\sim Mult\left(\beta_{z_{n}}\right)$ 
\end{enumerate}
\item For each ICD-9 code from the root of the hierarchy and recursively
descending the tree:

\begin{enumerate}
\item Draw a response variable $y_{i}\mid\bar{z},\eta_{i},y_{parent}\sim\Phi(\eta_{i}^{T}\bar{z},y_{parent})$
where $\bar{z}=N^{-1}\sum_{n=1}^{N}z_{n}$ and $\Phi$ refers to a
conditional probit model.
\end{enumerate}
\end{enumerate}
\end{enumerate}
We will employ a data augmentation scheme with auxiliary variables
$a_{i}$ in the probit model where:

\begin{equation}
y_{i}\sim\begin{cases}
\begin{array}{c}
1,\\
0,\end{array} & \begin{array}{c}
a_{i}>0\; and\; y_{parent}=1\\
otherwise\end{array}\end{cases}\end{equation}


\begin{equation}
a_{i}\sim\mathcal{N}\left(\eta_{i}^{T}\bar{z},1\right)\end{equation}



\subsection{Posterior Inference}

Given an observation of a set of ICD-9 codes and a document, the posterior
distribution for the latent variables is given by Equation 4.

The denominator for this distribution is the marginal probability
of the data and cannot be solved in closed form. This is often the
case in evaluating posterior distributions of non-trivial probabilistic
models. We will appeal to one of the common methods for approximating
posterior distributions in the face of intractable normalization factors:
Markov chain Monte Carlo (MCMC) sampling. Since in this model it is
possible to sample from the conditional distributions for all variables
we will use the Gibbs sampling algorithm to obtain an approximation
to this posterior. In particular, we will derive a collapsed Gibbs
sampler for the supervised topic model.


\subsection{Rao-Blackwellization}

To derive the Gibbs sampler in general, we integrate over the parameters
$\theta$ and $\phi_{1:K}$ resulting in the collapsed joint distribution
shown in Equations 5-8.

\begin{landscape} \centering % optional, probably makes it look better to have it centered on the page


\begin{small}

\begin{equation}
p\left(\theta,z_{1:N}\mid w_{1:N},y_{1:I},\phi_{1:K},\eta_{1:I},\alpha,\beta,\mu,\xi\right)=\frac{p\left(\theta\mid\alpha\right)\left(\prod_{n=1}^{N}p\left(z_{n}\mid\theta\right)p\left(w_{n}\mid z_{n},\phi_{1:K}\right)\right)\left(\prod_{k=1}^{K}p\left(\phi_{k}\mid\beta\right)\right)\left(\prod_{i=1}^{I}p\left(y_{i}\mid z_{1:N},\eta_{i},\xi\right)p\left(\eta_{i}\mid\mu\right)\right)}{\int_{\theta}p\left(\theta\mid\alpha\right)\sum_{k=1}^{K}\left(\prod_{n=1}^{N}p\left(z_{n}\mid\theta\right)p\left(w_{n}\mid z_{n},\phi_{1:K}\right)\right)\left(\prod_{k=1}^{K}p\left(\phi_{k}\mid\beta\right)\right)\left(\prod_{i=1}^{I}p\left(y_{i}\mid z_{1:N},\eta_{i},\xi\right)p\left(\eta_{i}\mid\mu\right)\right)d\theta}\end{equation}


\begin{equation}
\intop_{\theta}\intop_{\phi_{1:K}}p\left(\mathbf{Y},\mathbf{w},\mathbf{z},\mathbf{\theta},\mathbf{\phi},\mathbf{\eta},\alpha,\beta,\mu,\xi\right)d\theta d\phi_{1:K}=\intop_{\theta}\intop_{\phi_{1:K}}\prod_{k=1}^{K}p\left(\phi_{k};\beta\right)\prod_{m=1}^{M}\left\{ p\left(\theta_{m};\alpha\right)\prod_{n=1}^{N}\left[p\left(z_{m,n}\mid\theta_{m}\right)p\left(w_{m,n}\mid\phi_{z_{m,n}}\right)\right]\prod_{i=1}^{I}p\left(y_{m,i}\mid z_{m,1:N},\eta_{i},\xi\right)\right\} \prod_{i=1}^{I}p\left(\eta_{i};\mu\right)d\theta d\phi_{1:K}\end{equation}


\begin{equation}
p\left(\mathbf{Y},\mathbf{w},\mathbf{z},\mathbf{\eta},\alpha,\beta,\mu,\xi\right)=\prod_{i=1}^{I}\left[p\left(\eta_{i};\mu\right)\prod_{m=1}^{M}p\left(y_{m,i}\mid z_{m,1:N},\eta_{i},\xi\right)\right]\intop_{\phi_{1:K}}\prod_{k=1}^{K}p\left(\phi_{k};\beta\right)\prod_{m=1}^{M}\prod_{n=1}^{N}p\left(w_{m,n}\mid\phi_{z_{m,n}}\right)d\phi_{1:K}\intop_{\theta}\prod_{m=1}^{M}p\left(\theta_{m};\alpha\right)\prod_{n=1}^{N}p\left(z_{m,n}\mid\theta_{m}\right)d\theta\end{equation}


\begin{equation}
=\prod_{i=1}^{I}\left[p\left(\eta_{i};\mu\right)\prod_{m=1}^{M}p\left(y_{m,i}\mid z_{m,1:N},\eta_{i},\xi\right)\right]\prod_{k=1}^{K}\frac{\Gamma\left(\sum_{v=1}^{V}\beta_{v}\right)}{\prod_{v=1}^{V}\Gamma\left(\beta_{v}\right)}\frac{\prod_{v=1}^{V}\Gamma\left(n_{\left(\bullet\right),v}^{k}+\beta_{v}\right)}{\Gamma\left(\sum_{v=1}^{V}n_{\left(\bullet\right),v}^{k}+\beta_{v}\right)}\prod_{m=1}^{M}\frac{\Gamma\left(\sum_{k=1}^{K}\alpha_{k}\right)}{\prod_{k=1}^{K}\Gamma\left(\alpha_{k}\right)}\frac{\prod_{k=1}^{K}\Gamma\left(n_{m,\left(\bullet\right)}^{k}+\alpha_{k}\right)}{\Gamma\left(\sum_{k=1}^{K}n_{m,\left(\bullet\right)}^{k}+\alpha_{k}\right)}\end{equation}


\begin{equation}
=\prod_{i=1}^{I}\left[\mathcal{N}\left(\eta_{i}\mid\mu,1\right)\prod_{m=1}^{M}h\left(y\right)\exp\left\{ \left(\eta_{i}^{T}\bar{z}\right)y-A\left(\eta_{i}^{T}\bar{z}\right)\right\} \right]\prod_{k=1}^{K}\frac{\Gamma\left(\sum_{v=1}^{V}\beta_{v}\right)}{\prod_{v=1}^{V}\Gamma\left(\beta_{v}\right)}\frac{\prod_{v=1}^{V}\Gamma\left(n_{\left(\bullet\right),v}^{k}+\beta_{v}\right)}{\Gamma\left(\sum_{v=1}^{V}n_{\left(\bullet\right),v}^{k}+\beta_{v}\right)}\prod_{m=1}^{M}\frac{\Gamma\left(\sum_{k=1}^{K}\alpha_{k}\right)}{\prod_{k=1}^{K}\Gamma\left(\alpha_{k}\right)}\frac{\prod_{k=1}^{K}\Gamma\left(n_{m,\left(\bullet\right)}^{k}+\alpha_{k}\right)}{\Gamma\left(\sum_{k=1}^{K}n_{m,\left(\bullet\right)}^{k}+\alpha_{k}\right)}\end{equation}


\end{small} \end{landscape}


\subsection{Gibbs Sampling}

To derive the Gibbs sampler we evaluate the individual conditional
probability distributions for all unobserved variables.


\subsubsection{$p\left(z_{m,n}\mid\mathbf{z_{-\left(m,n\right)}},\mathbf{Y},\mathbf{w},\mathbf{\eta},\alpha,\beta,\mu,\xi\right)$}

For the purposes of sampling, we will be able to derive a representation
of the joint distribution isolating a particular latent variable,
$z$, for a word instance, $n$, in a document instance, $m$. The
conditional probability with respect to this latent variable is proportional
to the joint distribution up to a constant.

\begin{equation}
p\left(z_{m,n}\mid\mathbf{z_{-\left(m,n\right)}},\mathbf{Y},\mathbf{w},\mathbf{\eta},\alpha,\beta,\mu,\xi\right)\propto p\left(z_{m,n},\mathbf{z_{-\left(m,n\right)}},\mathbf{Y},\mathbf{w},\mathbf{\eta},\alpha,\beta,\mu,\xi\right)\end{equation}


Due to the factorization of this model we can rewrite the joint distribution
as the following:

\begin{equation}
\propto\prod_{i=1}^{I}\left[p\left(\eta_{i};\mu\right)p\left(y_{m,i}\mid z_{m,1:N},\eta_{i}\xi\right)\right]p\left(z_{m,n},\mathbf{z_{-\left(m,n\right)}},\mathbf{w},\mathbf{\eta},\alpha,\beta,\mu\right)\end{equation}


We isolate only terms that depend on $z_{m,n}$ and absorb all other
constant terms into the normalization constant \citep{Griffiths04}.

\begin{equation}
\propto\prod_{i=1}^{I}\left[p\left(y_{m,i}\mid z_{m,1:N},\eta_{i},\xi\right)\right]\left(n_{m,\left(\bullet\right)}^{k,-\left(m,n\right)}+\alpha_{k}\right)\frac{n_{\left(\cdot\right),w_{m,n}}^{k,-\left(m,n\right)}+\beta_{k,w_{m,n}}}{\sum_{v=1}^{V}\left(n_{\left(\cdot\right),w_{m,n}}^{k,-\left(m,n\right)}+\beta_{k,v}\right)}\end{equation}


Here, $n_{m,v}^{k,-\left(m,n\right)}$ represents the count of word
$v$ in document $m$ assigned to topic $k$ omitting the $(m,n)^{th}$
word count. For exponential family distributions, the normalization
contant, $h(y_{m,i})$, does not depend on $z_{m,n}$.

\begin{equation}
\propto\exp\left\{ \sum_{i=1}^{I}\left(\eta_{i}^{T}\bar{z}\right)y_{m,i}-A\left(\eta_{i}^{T}\bar{z}\right)\right\} \left(n_{m,\left(\bullet\right)}^{k,-\left(m,n\right)}+\alpha_{k}\right)\frac{n_{\left(\bullet\right),w_{m,n}}^{k,-\left(m,n\right)}+\beta_{k,w_{m,n}}}{\sum_{v=1}^{V}\left(n_{\left(\bullet\right),w_{m,n}}^{k,-\left(m,n\right)}+\beta_{k,v}\right)}\label{eq:z_likelihood}\end{equation}


Given this expression, $p\left(z_{m,n}\mid\mathbf{z_{-\left(m,n\right)}},\mathbf{Y},\mathbf{w},\mathbf{\eta},\alpha,\beta,\mu\right)$
can be sampled through enumeration as seen in Equation 13. In the
case of probit regression, the expression for \ref{eq:z_likelihood}
evaluates to Equation 14. Equivalently we can parameterize the model
with an auxiliary variable $a_{i}$, resulting in Equation 15.

%\begin{equation}
%p\left(z_{m,n}\mid\mathbf{z_{-\left(m,n\right)}},\mathbf{Y},\mathbf{w},\mathbf{\eta},\alpha,\beta,\mu\right)=\frac{\exp\left\{ \sum_{i=1}^{I}\left(\eta_{i}^{T}\bar{z}\right)y_{m,i}-A\left(\eta_{i}^{T}\bar{z}\right)\right\} \left(n_{m,\left(\bullet\right)}^{k,-\left(m,n\right)}+\alpha_{k}\right)\frac{n_{\left(\bullet\right),w_{m,n}}^{k,-\left(m,n\right)}+\beta_{k,w_{m,n}}}{\sum_{v=1}^{V}\left(n_{\left(\bullet\right),w_{m,n}}^{k,-\left(m,n\right)}+\beta_{k,v}\right)}}{\sum_{k=1}^{K}\exp\left\{ \sum_{i=1}^{I}\left(\eta_{i}^{T}\bar{z}\right)y_{m,i}-A\left(\eta_{i}^{T}\bar{z}\right)\right\} \left(n_{m,\left(\bullet\right)}^{k,-\left(m,n\right)}+\alpha_{k}\right)\frac{n_{\left(\bullet\right),w_{m,n}}^{k,-\left(m,n\right)}+\beta_{k,w_{m,n}}}{\sum_{v=1}^{V}\left(n_{\left(\bullet\right),w_{m,n}}^{k,-\left(m,n\right)}+\beta_{k,v}\right)}}\end{equation}


%\begin{equation}
%p\left(z_{m,n}\mid\mathbf{z_{-\left(m,n\right)}},\mathbf{w},\mathbf{\eta},\alpha,\beta,\mu\right)\propto\prod_{i=1}^{I}\left[\Phi\left(\eta_{i}^{T}\bar{z}\right)^{y_{m,i}}\left(1-\Phi\left(\eta_{i}^{T}\bar{z}\right)\right)^{1-y_{m,i}}\right]\left(n_{m,\left(\bullet\right)}^{k,-\left(m,n\right)}+\alpha_{k}\right)\frac{n_{\left(\bullet\right),w_{m,n}}^{k,-\left(m,n\right)}+\beta_{k,w_{m,n}}}{\sum_{v=1}^{V}\left(n_{\left(\bullet\right),w_{m,n}}^{k,-\left(m,n\right)}+\beta_{k,v}\right)}\end{equation}


%\begin{equation}
%p\left(z_{m,n}\mid\mathbf{z_{-\left(m,n\right)}},\mathbf{w},\mathbf{\eta},\mathbf{a},\alpha,\beta,\mu\right)\propto\prod_{i=1}^{I}\left[\delta\left(sign\left(a_{m,i}\right)=y_{m,i}\right)\mathcal{N}\left(a_{m,i}\mid\eta_{i}^{T}\bar{z,}1\right)\right]\left(n_{m,\left(\bullet\right)}^{k,-\left(m,n\right)}+\alpha_{k}\right)\frac{n_{\left(\bullet\right),w_{m,n}}^{k,-\left(m,n\right)}+\beta_{k,w_{m,n}}}{\sum_{v=1}^{V}\left(n_{\left(\bullet\right),w_{m,n}}^{k,-\left(m,n\right)}+\beta_{k,v}\right)}\end{equation}


\begin{landscape}

\centering % optional, probably makes it look better to have it centered on the page
\begin{equation}
p\left(z_{m,n}\mid\mathbf{z_{-\left(m,n\right)}},\mathbf{Y},\mathbf{w},\mathbf{\eta},\alpha,\beta,\mu\right)=\frac{\exp\left\{ \sum_{i=1}^{I}\left(\eta_{i}^{T}\bar{z}\right)y_{m,i}-A\left(\eta_{i}^{T}\bar{z}\right)\right\} \left(n_{m,\left(\bullet\right)}^{k,-\left(m,n\right)}+\alpha_{k}\right)\frac{n_{\left(\bullet\right),w_{m,n}}^{k,-\left(m,n\right)}+\beta_{k,w_{m,n}}}{\sum_{v=1}^{V}\left(n_{\left(\bullet\right),w_{m,n}}^{k,-\left(m,n\right)}+\beta_{k,v}\right)}}{\sum_{k=1}^{K}\exp\left\{ \sum_{i=1}^{I}\left(\eta_{i}^{T}\bar{z}\right)y_{m,i}-A\left(\eta_{i}^{T}\bar{z}\right)\right\} \left(n_{m,\left(\bullet\right)}^{k,-\left(m,n\right)}+\alpha_{k}\right)\frac{n_{\left(\bullet\right),w_{m,n}}^{k,-\left(m,n\right)}+\beta_{k,w_{m,n}}}{\sum_{v=1}^{V}\left(n_{\left(\bullet\right),w_{m,n}}^{k,-\left(m,n\right)}+\beta_{k,v}\right)}}\end{equation}


\begin{equation}
p\left(z_{m,n}\mid\mathbf{z_{-\left(m,n\right)}},\mathbf{w},\mathbf{\eta},\alpha,\beta,\mu\right)\propto\prod_{i=1}^{I}\left[\Phi\left(\eta_{i}^{T}\bar{z}\right)^{y_{m,i}}\left(1-\Phi\left(\eta_{i}^{T}\bar{z}\right)\right)^{1-y_{m,i}}\right]\left(n_{m,\left(\bullet\right)}^{k,-\left(m,n\right)}+\alpha_{k}\right)\frac{n_{\left(\bullet\right),w_{m,n}}^{k,-\left(m,n\right)}+\beta_{k,w_{m,n}}}{\sum_{v=1}^{V}\left(n_{\left(\bullet\right),w_{m,n}}^{k,-\left(m,n\right)}+\beta_{k,v}\right)}\end{equation}


\begin{equation}
p\left(z_{m,n}\mid\mathbf{z_{-\left(m,n\right)}},\mathbf{w},\mathbf{\eta},\mathbf{a},\alpha,\beta,\mu\right)\propto\prod_{i=1}^{I}\left[\delta\left(sign\left(a_{m,i}\right)=2y_{m,i}-1\right)\mathcal{N}\left(a_{m,i}\mid\eta_{i}^{T}\bar{z,}1\right)\right]\left(n_{m,\left(\bullet\right)}^{k,-\left(m,n\right)}+\alpha_{k}\right)\frac{n_{\left(\bullet\right),w_{m,n}}^{k,-\left(m,n\right)}+\beta_{k,w_{m,n}}}{\sum_{v=1}^{V}\left(n_{\left(\bullet\right),w_{m,n}}^{k,-\left(m,n\right)}+\beta_{k,v}\right)}\end{equation}


\end{landscape}


\subsubsection{$p\left(\eta_{i}\mid\mathbf{z},\mathbf{Y},\mu\right)$ or $p\left(\eta_{i}\mid\mathbf{z},\mathbf{a},\mu\right)$in
the augmented probit regression model}

Given that $\eta_{i}$ and $a_{m,i}$ are distributed normally, this
posterior distribution is also normal. In the case for general exponential
family distributions, $\eta_{i}$ can remain a parameter without a
prior, fit with maximimum likelihood in the usual fashion.

\begin{equation}
p\left(\eta_{i}\mid\mathbf{z},\mathbf{a},\mu\right)=\mathcal{N}\left(\eta_{i}\mid\hat{\mathbf{\mu}_{i}},\hat{\mathbf{S}_{i}}\right)\end{equation}


\[
\hat{\mathbf{\mu}_{i}}=\hat{\mathbf{S}_{i}}\bar{\mathbf{Z}}^{T}\mathbf{a}_{\left(\bullet\right),i}\]


\[
\hat{\mathbf{S}_{i}}^{-1}=\mathbf{I}+\bar{\mathbf{Z}}^{T}\bar{\mathbf{Z}}\]



\subsubsection{$p\left(a_{m,i}\mid\mathbf{z},\mathbf{Y},\mathbf{\eta}\right)$ in
the augmented probit regression model}

In the augmented probit regression model, the posterior distribution
of $a_{i}$ is distributed according to a truncated normal distribution.

\[
p\left(a_{m,i}\mid\mathbf{z},\mathbf{Y},\mathbf{\eta}\right)=trunc\mathcal{N}\left(a_{m,i}\mid\eta_{i}^{T}\bar{z,}1,y_{m,i}\right)\]



\subsubsection{\textmd{$p\left(y_{m,i}\mid\mathbf{\eta},\mathbf{a},\xi\right)$}}

In our model, response variables are not always observed and are treated
as latent and sampled where appropriate. There are two factors influencing
predictions of the response variable, $y_{m,i}$. There is an undirected
model enforcing the aforementioned constraints and providing a prior
and there is the probit regression.

\begin{equation}
p\left(y_{m,i}\mid\mathbf{\eta},\mathbf{a},\xi\right)\propto\psi\left(\mathbf{Y}\right)\delta\left(sign\left(a_{m,i}\right)=y_{m,i}\right)\mathcal{N}\left(a_{m,i}\mid\eta_{i}^{T}\bar{z,}1\right)\end{equation}


\begin{equation}
p\left(y_{m,i}\mid\mathbf{\eta},\mathbf{a},\xi\right)\propto\psi\left(\mathbf{Y}\right)trunc\mathcal{N}\left(a_{m,i}\mid\eta_{i}^{T}\bar{z,}1,y_{m,i}\right)\end{equation}


Again, this conditional distribution can be evaluated through enumerations
and normalization.

\begin{equation}
p\left(y_{m,i}\mid\mathbf{\eta},\mathbf{a},\xi\right)=\frac{\psi\left(\mathbf{Y}\right)trunc\mathcal{N}\left(a_{m,i}\mid\eta_{i}^{T}\bar{z,}1,y_{m,i}\right)}{\sum_{y_{m,i}}\psi\left(\mathbf{Y}\right)trunc\mathcal{N}\left(a_{m,i}\mid\eta_{i}^{T}\bar{z,}1,y_{m,i}\right)}\end{equation}



\section{Results}

%We ran the Gibbs sampler for two days on 500 documents with 20 topics to learn the sLDA model which we then followed by a prediction of ICD-9 codes for 100 documents. The prediction resulted in 38 documents with at least 1 ICD-9 code assignment while some had over 200 assignments.


%\begin{figure}[h]%tbp] %  figure placement: here, top, bottom, or page
%   \centering
%   \includegraphics[width=4in]{number_of_codes_histogram} 
%   \caption{distribution of ICD-9 code assignments over all of the 100 documents predicted}
%   \label{fig:example}
%\end{figure}


%The accuracy of the prediction was tested against the actual ICD-9 codes in the EHR record for each discharge summary.  Different statistical measures were applied to asses the predictions.


%\begin{figure}[h]%tbp] %  figure placement: here, top, bottom, or page
%   \centering
%   \includegraphics[width=4in]{signal_detection} 
%   \caption{sensitivity, specificity, precision, recall and F-measure.}
%   \label{fig:example}
%\end{figure}


%Although the sensitivty is not as high as we would have hoped, it is important to note that the sLDA was run on a very limited sample of documents (500) due to time constraints.  Additionally, the “gold standard” data used are actual hospital results which are very noisy given the incompleteness and introduction of human error.



\section{Conclusion}

%We have proposed a new method for prediction of ICD-9 codes given a medical discharge summary note.  Our method explicitly takes advantage of the ICD-9 hierarchical IS-A relationships as well as the word distributions in each document.  We used Gibbs sampling for a supervised latent dirichlet allocation model to learn a distribution of words over twenty topics and their appropriate ICD-9 codes (response variable). The model was then tested on 100 discharge summaries for which the ICD-9 diagnoses were known.  The amount of ICD-9 codes produced for some discharge summaries was quite high because of the way the model worked with the IS-A hierarchy, highlighting the completeness of ICD-9 coding when using this sLDA model.  Overall, the model had extremely good specificity while the sensitivity was lower than anticipated.  Given our limited set of documents, and very noisy gold standard it would not be accurate to discount the use of sLDA in medical document classification and automatic coding.
%In future work, we plan to significantly expand the set of documents used to train the model (up to 15,000), as well as run several different iterations to learn the optimal topic number.  


%A more automated approach for coding patient visits with the ICD-9 hierarchy has great promise to  reduce health care costs and increase completeness of medical coded data.  We hope that this implementation of supervised LDA will serve as preliminary work for a much larger study with more documents in the future.


\begin{small} \bibliographystyle{plainnat} \bibliographystyle{plainnat}
\bibliography{refs}
 \end{small} 
\end{document}

