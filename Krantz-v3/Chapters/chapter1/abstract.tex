We introduce hierarchically supervised latent Dirichlet allocation (HSLDA), a
model for hierarchically and multiply labeled bag-of-word data.  Examples of
such data include web pages and their placement in directories, product
descriptions and associated categories from product hierarchies, and free-text
clinical records and their assigned diagnosis codes. Out-of-sample label
prediction is the primary goal of this work, but improved lower-dimensional
representations of the bag-of-word data are also of interest.
%We define a model that uses probit regressors on a conditionally dependent
%label hierarchy tied to latent Dirichlet allocation (LDA). 
We demonstrate HSLDA on large-scale data from clinical document labeling and
retail product categorization tasks. We show that leveraging the structure from
hierarchical labels improves out-of-sample label prediction substantially when
compared to models that do not. %Furthermore, we show evidence
%that the resulting HSLDA topics are more descriptive of the underlying data
%than sLDA topics, which ignore the label hierarchy.

%Topic models are unsupervised models well suited to data exploration and information retrieval.  Supervised topic models use side-information, labels and the like, to improve the learned representations.

%The benefits of supervision in topic modeling 
%Current medical record keeping technology relies heavily upon human
%capacity to effectively summarize and infer information from free-text
%physician notes. We propose a novel method to suggest diagnostic code
%assignment for patient visits, based upon narrative medical notes.
%We applied a supervised latent Dirichlet allocation model to a corpora
%of free-text medical notes from NewYork - Presbyterian Hospital to
%infer a set of specific ICD-9 codes for each patient note. Evaluation
%of the predictions were conducted by comparison to a gold-standard
%set of ICD-9s assigned to a set of patient notes. 
