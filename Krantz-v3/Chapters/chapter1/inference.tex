
\label{sec:inference}
Our inference goal  is to obtain a representation of the posterior distribution
of the latent variables in the model.  This posterior distribution can then be used for predictive inference of labels for held-out documents among other things.  Unfortunately the posterior distribution we seek does not have a simple analytic form from which exact samples can be drawn.
This is usually the case for posterior distributions of non-trivial
probabilistic models and suggests approximating the posterior distribution by sampling.

In this section we derive the conditional distributions required to sample from the HSLDA posterior distribution using Markov chain Monte Carlo.  %A very brief review of Markov chain Monte Carlo can be found in Section \ref{sec:mcmc} at the end of this chapter.  
%
The HSLDA sampler, like the collapsed Gibbs samplers for LDA \cite{Griffiths04}, is itself a collapsed Gibbs sampler in which all of the latent variables that can be analytically marginalized are.  Among others, the topic distributions $\boldsymbol{\phi}_{1:K}$
and document specific topic assignment distributions $\boldsymbol{\theta}_{1:D}$ are analytically marginalized prior to deriving the following conditional distributions for sampling.  

  It will usually be the
case that values $y_{l,d}$ will not be known for all labels $l \in \mathcal{L}$ in the space of possible labels.  Values for $y_{l,d}$ that are enforced by label constraints and observed labels are set to their constrained values prior to inference and treated as observed.  We will define $\mathcal{L}_{d}$ to be the subset
of labels which have been observed (or observed via filling in from constraints) for document $d$.   Marginalizing the probit regression auxiliary variables $a_{l^{\prime},d}$ and $y_{l^{\prime},d}$
for $l^{\prime}\in\mathcal{L}\backslash\mathcal{L}_{d}$ is simple in the is-a hierarchy case because they can simply be ignored. 
%We can also integrate out the parameters $\boldsymbol{\phi}_{1:K}$
%and $\boldsymbol{\theta}_{1:D}$ as in \cite{Griffiths04}.
 The remaining latent variables (those that are not collapsed out) are the topic indicators $\mathbf{z}=\{z_{1:N_{d},d}\}_{d=1,\ldots,D},$ the probit regression parameters $\boldsymbol\eta=\{\boldsymbol\eta_{l}\}_{l\in\mathcal{L}},$ the auxiliary variables $\mathbf{a}=\{a_{l^{\prime},d}\}_{l^{\prime}\in\mathcal{L}_{d},d=1,\ldots,D},$ the global topic proportions $\boldsymbol\beta$, and the concentration parameters $\alpha,\alpha^{\prime}$
and $\gamma$.


%We will appeal to one of the common methods
%for approximating posterior distributions in the face of intractable
%normalization factors: Markov chain Monte Carlo (MCMC) sampling. 
%Since
% the conditional distributions
%for all variables are relatively simple (enumeration and, in the case of the profit regression coefficients, conjugate) we will use Gibbs sampling. %In particular, we will derive a collapsed Gibbs sampler.
%Given an observation of a set of observed labels and a document, the posterior distribution for the latent variables is
%\begin{equation}
%p\left(\theta,z_{1:N},\phi_{1:K},\boldsymbol\eta_{i_{l,c}\in\mathcal{I}},a_{i_{l,c}\in\mathcal{I}},\boldsymbol\beta,\alpha,\alpha',\gamma\mid w_{1:N},y_{i_{l,c}\in\mathcal{I}};\sigma,\lambda\right)\label{eq:Posterior}\end{equation}
%\[
%=\frac{p\left(\theta,z_{1:N},\phi_{1:K},\boldsymbol\eta_{i_{l,c}\in\mathcal{I}},a_{i_{l,c}\in\mathcal{I}},\boldsymbol\beta,\alpha,\alpha',\gamma,w_{1:N},y_{i_{l,c}\in\mathcal{I}};\sigma,\lambda\right)}{\int_{\theta,\phi,a,\boldsymbol\eta,\alpha,\alpha',\boldsymbol\beta,\gamma}\sum_{z}p\left(\theta,z_{1:N},\phi_{1:K},\boldsymbol\eta_{i_{l,c}\in\mathcal{I}},a_{i_{l,c}\in\mathcal{I}},\boldsymbol\beta,\alpha,\alpha',\gamma,w_{1:N},y_{i_{l,c}\in\mathcal{I}};\sigma,\lambda\right)}\]
%
%

\subsection{Gibbs Sampler}
 Let $\mathbf{a}$ be the set of all probit regression auxiliary variables, $\mathbf{w}$  the set of all words, $\boldsymbol\eta$  the set of all regression coefficients, and  $\mathbf{z}\backslash z_{n,d}$  the set $\mathbf{z}$ with element $z_{n,d}$ removed.  %The conditional posterior distribution of the latent topic indicators is
 %We use the notation $\mathbf{z}_{-(n,d)}$ to
%denote $\mathbf{z}_d\backslash z_{n,d}$. %, marginalizing
%$\theta_{1:D}$ and $\phi_{1:K}$. For details regarding collapsing
%in LDA models see \cite{Griffiths04}. 



%\subsection{Sampling latent topic indicators}
%$p(z_{n,d}\mid\mathbf{z}_{-(n,d)},\mathbf{a},\mathbf{w},\mathbf{\boldsymbol\eta},\alpha,\boldsymbol\beta,\gamma)$}

First we consider the conditional distribution of $z_{n,d}$ (the assignment variable
for each word $n=1,\ldots,N_{d}$ in documents $d=1,\ldots,D$). 
%The
%conditional distribution does not include $\theta_{1:D}$ and $\phi_{1:K}$
%because they have been integrated out as in the collapsed Gibbs sampler
%\cite{Griffiths04}. 
%The conditional distribution of $z_{n,d}$ is
%proportional to the joint distribution of its markov blanket. \begin{equation}
%p\left(z_{d,n}\mid\mathbf{z_{-\left(d,n\right)}},\mathbf{a},\mathbf{w},\mathbf{\boldsymbol\eta},\alpha,\boldsymbol\beta,\gamma\right)\propto p\left(z_{d,n},\mathbf{z_{-\left(d,n\right)}},\mathbf{a},\mathbf{w},\mathbf{\boldsymbol\eta},\alpha,\boldsymbol\beta,\gamma\right)\end{equation}
Following the factorization of the model (refer again to Figure~\ref{fig:graphical_model}), we can write
\begin{eqnarray*}
\lefteqn{p\left(z_{n,d}\mid\mathbf{z}_d\backslash z_{n,d},\mathbf{a},\mathbf{w},\mathbf{\boldsymbol\eta},\alpha,\boldsymbol\beta,\gamma\right)}\\
&\propto&\prod_{l\in\mathcal{L}_{d}}p\left(a_{l,d}\mid\mathbf{z},\boldsymbol\eta_{l}\right)p\left(z_{n,d}\ |\ \mathbf{z}_d\backslash z_{n,d},\mathbf{a},\mathbf{w},\alpha,\boldsymbol\beta,\gamma\right).\end{eqnarray*}
 The product is only over the subset of labels $\mathcal{L}_{d}$
which have been observed for document $d$. By isolating terms that
depend on $z_{n,d}$ and absorbing all other terms into a normalizing
constant as in \cite{Griffiths04} we find 
\begin{eqnarray}
\lefteqn{p\left(z_{n,d}=k\mid\mathbf{z}\backslash z_{n,d},\mathbf{a},\mathbf{w},\mathbf{\boldsymbol\eta},\alpha,\boldsymbol\beta,\gamma\right)\propto} \label{eq:z-likelihood} \\
 & \hspace{2cm}\left(c_{\left(\cdot\right),d}^{k,-\left(n,d\right)}+\alpha\boldsymbol\beta_{k}\right)\frac{c_{w_{n,d},\left(\cdot\right)}^{k,-\left(n,d\right)}+\gamma}{\left(c_{\left(\cdot\right),\left(\cdot\right)}^{k,-\left(n,d\right)}+V\gamma\right)}\prod_{l\in\mathcal{L}_{d}}\exp\left\{ -\frac{\left(\bar{\mathbf{z}}_{d}^{T}\boldsymbol\eta_{l}-a_{l,d}\right)^{2}}{2}\right\}\nonumber\end{eqnarray}
 where $c_{v,d}^{k,-\left(n,d\right)}$ is the number
of words of type $v$ in document $d$ assigned to topic $k$ omitting
the $n$th word of document $d$. The subscript $(\cdot)$'s
indicate to sum over the range of the replaced variable, i.e.~$ {c_{w_{n,d},\left(\cdot\right)}^{k,-\left(n,d\right)}} = \sum_d {c_{w_{n,d},d}^{k,-\left(n,d\right)}}$.  Here $\mathcal{L}_{d}$ is the set of labels which are observed for document $d$.  We sample from \eqref{eq:z-likelihood}  by first enumerating $z_{n,d}$ and then normalizing.
%
%Given Equation \ref{eq:z-likelihood}, $p\left(z_{d,n}\mid\mathbf{z}_{-\left(d,n\right)},\mathbf{a},\mathbf{w},\mathbf{\boldsymbol\eta},\alpha,\boldsymbol\beta,\gamma\right)$
%can be sampled through enumeration. 

The conditional posterior distribution of the regression coefficients is given by 
\begin{equation}
p(\boldsymbol\eta_{l}\mid\mathbf{z},\mathbf{a},\sigma) = \mathcal{N}(\hat{\boldsymbol\mu}_{l},\hat{\mathbf{\Sigma}})\label{eqn:regression_param_conditional}
\end{equation}
$\boldsymbol\eta_{l}$ for $l\in\mathcal{L}$. Given that $\boldsymbol\eta_{l}$
and $a_{l,d}$ are distributed normally, the posterior distribution
of $\boldsymbol\eta_{l}$ is normally distributed with mean $\hat{\boldsymbol\mu}_{l}$
and covariance $\hat{\mathbf{\Sigma}}.$% such that % (probably not the right place for this) We evaluated the model over various values of $\sigma$ where $\sigma=\left\{ 0.01,0.1,0.25,1,2\right\} $.
where
\begin{equation*}
\hat{\boldsymbol\mu}_{l}  =  \hat{\mathbf{\Sigma}}\left(\mathbf{1}\frac{\mu}{\sigma}+\bar{\mathbf{Z}}^{T}\mathbf{a}_{l}\right) \qquad \hat{\mathbf{\Sigma}}^{-1}  =  \mathbf{I}\sigma^{-1}+\bar{\mathbf{Z}}^{T}\bar{\mathbf{Z}}
.\end{equation*}
Here $\bar{\mathbf{Z}}$ is a $D\times K$ matrix
such that row $d$ of $\mathbf{\bar{Z}}$ is $\bar{\mathbf{z}}_{d}$, and $\mathbf{a}_{l}=[a_{l,1},a_{l,2},\ldots,a_{l,D}]^{T}$.  The simplicity of this conditional distribution follows from the choice of probit regression  \cite{Albert_Chib_1993}; the specific form of the update is a standard result from Bayesian normal  linear regression \cite{gelmanbda04}. 
%p\left(\boldsymbol\eta_{i_{l,c}}\mid\mathbf{z}_{1:D},\mathbf{a};\sigma\right)=\mathcal{N}\left(\boldsymbol\eta_{i_{l,c}}\mid\hat{\mu}_{i},\hat{\Sigma}_{i}\right)\end{equation}
%\[
%\[
%\subsection{$p\left(a_{l,d}\mid\mathbf{z},\mathbf{Y},\mathbf{\boldsymbol\eta}\right)$}
%and \textmd{$p\left(y_{m,i}\mid\mathbf{a}\right)$}}
%The auxiliary variables $a_{l,d}$ must be sampled for documents $d=1,\ldots,D$
%and $l\in\mathcal{L}_{d}$. The conditional posterior distribution
%of 
It also is a standard probit regression result that the conditional posterior
distribution of $a_{l,d}$ is a truncated normal
distribution~\cite{Albert_Chib_1993} (see also Section \ref{sec:probit_review}). 
\begin{eqnarray*}
\lefteqn{p\left(a_{l,d}\mid\mathbf{z},\mathbf{Y},\mathbf{\boldsymbol\eta}\right)}\\&\propto&
\begin{cases}
\exp\left\{ -\frac{1}{2}\left(a_{l,d}-\boldsymbol\eta_{l}^{T}\mathbf{\bar{z}}_{d}\right)\right\} \mathbb{I}\left(a_{l,d}y_{l,d}>0\right)\mathbb{I}(a_{l,d}<0) , & y_{\mathrm{pa}(l),d}=-1\\
\exp\left\{ -\frac{1}{2}\left(a_{l,d}-\boldsymbol\eta_{l}^{T}\mathbf{\bar{z}}_{d}\right)\right\} \mathbb{I}\left(a_{l,d}y_{l,d}>0\right) , & y_{\mathrm{pa}(l),d}=1\end{cases}\label{eqn:a_l_d}\end{eqnarray*}

%\begin{equation}
%p\left(a_{l,d,}\mid\mathbf{z},\mathbf{Y},\mathbf{\boldsymbol\eta}\right)\propto\frac{1}{\sqrt{2\pi}}\exp\left\{ -\frac{1}{2}\left(a_{l,d}-\boldsymbol\eta_{l}^{T}\mathbf{\bar{z}}_{d}\right)\right\} \mathbb{I}\left(a_{l,d}y_{l,d}>0\right).\label{eqn:a_l_d}\end{equation}
% This conditional distribution can be sampled using an inverse CDF
%method. %However, if $y_{d,i_{l,c}}$ is unobserved then $a_{d,i_{l,c}}$must
%be sampled jointly with $y_{d,i_{l,c}}$ to ensure that the Markov
%chain is ergodic. Suppose that $a_{d,i_{l,c}}$ is sampled to have
%a negative value and $y_{d,i_{l,c}}$ is apporopriately sampled at
%-1. Although there exist valid states where $a_{d,i_{l,c}}>0$ and
%$y_{d,i_{l,c}}=1$, they will never be reached by such a Markov chain
%since $p\left(a_{d,i_{l,c}}<0\mid y_{d,i_{l,c}}=-1\right)=1$ and
%$p\left(y_{d,i_{l,c}}=-1\mid a_{d,i_{l,c}}<0\right)=1$. Therefore,
%to ensure ergodicity, $a_{d,i_{l,c}}$and $y_{d,i_{l,c}}$ must be
%sampled from the joint distribution as shown in Equation \ref{eq:Probit-Joint}.
%\begin{equation}
%p\left(a_{d,i_{l,c}},y_{d,i_{l,c}}\mid\mathbf{z},\mathbf{Y}_{-\left(d,i_{l,c}\right)},\mathbf{\boldsymbol\eta}\right)\propto p\left(y_{i_{l,c}}\mid\mathbf{a},\mathbf{y}_{-\left(l,c\right)}\right)p\left(a_{d,i_{l,c}}\mid\mathbf{z},\mathbf{Y},\mathbf{\boldsymbol\eta}\right)\end{equation}
%\begin{equation}
%p\left(y_{i_{l,c}}\mid\mathbf{a},\mathbf{y}_{-\left(l,c\right)}\right)=\delta\left(sign\left(a_{d,i_{l,c}}\right)=y_{i_{l,c}}\right)p\left(y_{i_{l,c}}\mid y_{parents_{l,c}}\right)\prod_{i_{\hat{l},\hat{c}}\in children_{l,c}}p\left(y_{i_{\hat{l},\hat{c}}}\mid y_{i_{l,c}}\right)\end{equation}
%\begin{equation}
%p\left(y_{i_{l,c}}=-1\mid y_{parent{}_{l,c}}\right)=\begin{cases}
%1, & y_{parent_{l,c}}=-1\\
%0.5, & y_{parent_{l,c}}=1\end{cases}\end{equation}
%\[
%p\left(a_{d,i_{l,c}},y_{d,i_{l,c}}\mid\mathbf{z},\mathbf{Y}_{-\left(d,i_{l,c}\right)},\mathbf{\boldsymbol\eta}\right)\]
%\begin{equation}
%=\begin{cases}
%\mathcal{N}\left(a_{d,i_{l,c}}\mid\bar{z}^{T}\boldsymbol\eta_{i_{l,c}},1\right)p\left(y_{d,i_{l,c}}\mid a_{d,i_{l,c}}\right), & y_{parent_{l,c}}=1,\forall y_{i_{\hat{l},\hat{c}}}\in y_{children_{l,c}},y_{i_{\hat{l},\hat{c}}}=-1\\
%trunc\mathcal{N}^{-}\left(a_{d,i_{l,c}}\mid\bar{z}^{T}\boldsymbol\eta_{i_{l,c}},1\right)\delta\left(y_{d,i_{l,c}}=-1\right), & y_{parent_{l,c}}=-1\\
%trunc\mathcal{N}^{+}\left(a_{d,i_{l,c}}\mid\bar{z}^{T}\boldsymbol\eta_{i_{l,c}},1\right)\delta\left(y_{d,i_{l,c}}=1\right), & \exists y_{i_{\hat{l},\hat{c}}}\in y_{children_{l,c}}\setminus y_{i_{\hat{l},\hat{c}}}=1\\
%0 & otherwise\end{cases}\end{equation}
%where $\mathbf{Y}_{-\left(d,i_{l,c}\right)}$ denotes all of the response
%variables excluding the response variable being sampled.
%
%
%
%\subsection{$p\left(\boldsymbol\beta\mid\mathbf{z},\alpha^{\prime},\alpha\right)$}

%Note that care must be taken to initialize the Gibbs sampler in a valid state.

HSLDA employs a hierarchical Dirichlet prior over topic assignments (i.e.,~$\boldsymbol\beta$ is estimated from data rather than fixed a priori).  This has been shown to improve the quality and stability of inferred topics \cite{WallachMiMc2009}. 
%This prior
%shares many features with the hierarchical Dirichlet process and inference
%over this distribution proceeds in a very similar fashion.
%
Sampling $\boldsymbol\beta$, the vector of global topic proportions,   can be done using the ``direct assignment''
method of \cite{TehJorBea2006} \begin{equation}
\boldsymbol\beta\mid\mathbf{z},\alpha^{\prime},\alpha\sim{\rm Dir}\left(m_{\left(\cdot\right),1}+\alpha^{\prime},m_{\left(\cdot\right),2}+\alpha^{\prime},\ldots,m_{\left(\cdot\right),K}+\alpha^{\prime}.\right)\end{equation}
Here $m_{d,k}$ are additional auxiliary variables that are introduced by the direct assignment method to sample the posterior distribution of $\boldsymbol\beta$.  Their conditional posterior distribution is sampled according to
 \begin{equation}
p\left(m_{d,k}=m\mid\mathbf{z},\mathbf{m}_{-\left(d,k\right)},\boldsymbol\beta\right)=\frac{\Gamma\left(\alpha\boldsymbol\beta_{k}\right)}{\Gamma\left(\alpha\boldsymbol\beta_{k}+c^k_{(\cdot),d}\right)}s\left(c^k_{(\cdot),d},m\right)\left(\alpha\boldsymbol\beta_{k}\right)^{m}\end{equation}
where $s\left(n,m\right)$ denotes Stirling numbers of the first
kind.
%\subsection{$p\left(\alpha\right)$, $p\left(\alpha'\right)$, $p\left(\gamma\right)$}
The hyperparameters $\alpha$, $\alpha^{\prime}$, and $\gamma$ are
%given broad ${\rm Gamma}(1,1000)$ prior distributions and 
sampled
using Metropolis-Hastings. \newline

 
It remains now to show that HSLDA works.  To do so we demonstrate results from modeling real-world datasets in the clinical and web
retail domains. These results provide evidence that the two views (text and labels) mutually benefit  multi-label
classification. %In fact, it will be shown that the hierarchical supervised model outperforms a
That is, modeling the joint  is better than learning 
topic models and hierarchical classifiers
independently. 
